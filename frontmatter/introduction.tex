\chapter{Introduction}
\label{introduction}

\section[The McDonald Family in South Africa]{The \mcdonald{} Family in \southAfrica}
\label{introduction: overview}

The \mcdonald{} name has a rich and varied history in \southAfrica{}, with many members of the \mcdonald{} family arriving through different avenues. Some came as soldiers, others as sailors, merchants, and among them were also the \settler s who embarked on the journey to start anew in the \capeColony{}.

Among the \settler s, two distinct \mcdonald{} families immigrated to \southAfrica{}. \mcdonaldPName{} travelled with \synnotWName{}'s party, while \mcdonaldJName{} came with \biggarAName’s party aboard the \weymouth{}. Each of these families has its own unique story and contributions to the history of \southAfrica{}. However, this book is dedicated solely to the lineage of \mcdonaldJName{} and \welchMName{}.

\section[British 1820 Settlers]{\settlersBritish{}}
\label{introduction: settlers}

The \settlersBritish{} played a pivotal role in the colonization of the \easternCape{}, \southAfrica{}. These settlers were primarily motivated by economic hardship in \britain{}, following the Napoleonic Wars, with the government encouraging emigration to reinforce the British presence in the region. These settlers embarked on a journey with the promise of land, opportunity, and a new life.

Upon arrival, the Settlers faced a harsh reality. Instead of fertile farmland, they were allocated land on the \easternCape{} frontier, a region fraught with tension between \xhosa{} communities and the British colonists. Letters from the Settlers to the \colonialOffice{} reveal the settlers' struggles, inadequate support, poor soil quality, and constant threat of conflict \autocite{eggsa:settlerStatement}.

One notable group was the party led by \biggarAFullNames{}, a former army officer who had fallen from grace due to financial misconduct. He saw the \capeColony{} as an opportunity to rebuild his reputation and life \autocite[45]{nash:1820}. From \citetitle{nash:1820} we learn that:
\begin{displayquote}
    Articles of Agreement were signed between \biggarASurname{} and his party at \portsmouth{} on \DTMDate{1819-12-13} before they embarked, all the men of the party were indentured to \biggarASurname{} for a period of three years, and were to receive food and clothing but no wages for their first year of service. In the second and third years they would be paid wages 'according to colonial practice'. Each man would be given 20 acres of land which he would be free to cultivate on Saturdays and Sundays, and at the end of the period of service he would receive title to it, although \biggarASurname{} was to retain 'Manorial Rights \autocite[45]{nash:1820}.
\end{displayquote}

His party sailed on the \weymouth{}, departing from \portsmouth{} on \DTMDate{1820-01-07}, and arriving in \algoaBay{} (modern-day \portElizabeth{}) on \DTMDate{1820-05-15} \autocite[45]{nash:1820}. After their arrival, \biggarASurname{} and his party were located at \drieFontein{} on the \brakRivier{}. With the single exception of \pollardGName{}, \biggarASurname{}'s labourers deserted or applied to be released from their engagements soon after they reached \albany{}. By July 1820 several of them were employed at \somersetFarm{}. By 1826 the party had disband completely and the location was granted to \biggarAFullNames{} \autocite[37]{tanner:2019}